\chapter{Operaciones numericas}

\textbf{MATLAB} puede operar como una calculadora: si el usuario escribe las ordenes apropiadas, los resultados aparecen en la ventana de comandos (Command Window). Observe que si se coloca $;$ (punto y coma) al final de la expresión el resultado no se escribe en la pantalla. Es capaz de realizar las operaciones aritméticas suma, resta, multiplicación, división y potenciación, con números (reales y complejos), con vectores (polinomios) y con matrices. Además, mediante la librería Symbolic Math Toolbox, se puede también operar con expresiones simbólicas.

\section{Operaciones aritméticas}

Las operaciones aritméticas con números son, quizás, las más sencillas que pueden efectuarse. Para ilustrar su realización, a continuación se muestran una serie de líneas que comienzan con $>>$, indicativo (o prompt) de la pantalla de comandos en una sesión de $Matlab$, seguido de una orden y del resultado que aparecerá inmediatamente en la pantalla si se ejecutara. Para comenzar...

\paragraph{Operar sin asignar el resultado a una variable:}si el usuario no ha asignado el resultado a una variable, \textbf{MATLAB} lo hace utilizando la
variable $ans$.
\begin{lstlisting}[language=Matlab]
>> 2 + 2 % y luego hacer ENTER
\end{lstlisting}

... y el resultado es...

\begin{lstlisting}[language=Matlab]
>> 2 + 7
ans = 9
\end{lstlisting}

\paragraph{Asignar el resultado a una variable:}si el calculo se asigna a una variable, este queda guardado en dicha variable:
\begin{lstlisting}[language=Matlab]
>> a = 2 + 7
a = 9
\end{lstlisting}

\paragraph{Conocer el valor de una variable:}para conocer el valor de una variable, solo debemos teclear el nombre de dicha variable:
\begin{lstlisting}[language=Matlab]
>> a
a = 9
\end{lstlisting}

\paragraph{Como evitar mostrar siempre el resultado:}si se añade un punto y coma ( ; ) al finalizar un renglón de la línea de comando, el programa no muestra la respuesta...
\begin{lstlisting}[language=Matlab]
>> j = 2 + 6;
\end{lstlisting}

... pero de igual forma la realiza, aunque no muestre el resultado:
\begin{lstlisting}[language=Matlab]
>> j
j = 8
\end{lstlisting}

\paragraph{Orden de las operaciones a evaluar:}las operaciones se evalúan por el orden usual de prioridad: primero las potencias, después las multiplicaciones y divisiones y, finalmente, las sumás y restas. Las operaciones de igual prioridad se evalúan de izquierda a derecha:
\begin{lstlisting}[language=Matlab]
>> 2/4*3
ans = 1.500
>> 2/(4*3)
ans = 0.1.667
\end{lstlisting}

\paragraph{Algunas funciones matemáticas:}las funciones matemáticas usuales están disponibles (llamadas built-in functions) y su sintaxis es la habitual en matemática. Por ejemplo:
\begin{lstlisting}[language=Matlab]
>> cos(pi)
ans = -1
\end{lstlisting}

La función exponencial es $exp$:
\begin{lstlisting}[language=Matlab]
>> exp(1)
ans = 2.7183
\end{lstlisting}


\paragraph{Variables con valor predeterminado:}$\pi$ es una variable con valor predeterminado que en el formato de números por defecto al iniciar la sesión de \textbf{MATLAB} es:
\begin{lstlisting}[language=Matlab]
>> pi
pi = 3.1416
\end{lstlisting}

Hay muchas otras variables con valor predeterminado, que se pierde si se les asigna otro valor. Por ejemplo, el epsilon de la máquina es $eps$:
\begin{lstlisting}[language=Matlab]
>> eps
eps = 2.2204e-16
\end{lstlisting}

\paragraph{Cantidad de decimales:}desde la línea de comando se puede controlar el número de decimales con que se muestra el valor de las variables o cálculos. Esto no determina la precisión con que se realizan los cálculos sino el formato con que se muestran en pantalla.
\begin{lstlisting}[language=Matlab]
>> 1/6
ans = 0.1667
>>
>> format long
>> 1/6
ans = 0.166666666666667
\end{lstlisting}

Se sugiere mirar la ayuda de format (en la línea de comando teclear help format ) para mirar la variedad de formatos de salida para números en la pantalla. Para retornar al modo de visualización por defecto (format short) teclear format en la línea de comando.

\paragraph{Conocer las variables que hemos usado:}en muchas oportunidades resulta de interés conocer las variables que están definidas hasta una determinada instancia de la sesión. Si usamos el comando $who$ veremos las variables usadas.
\begin{lstlisting}[language=Matlab]
>> who
a j
\end{lstlisting}

\paragraph{Dejar en cero una variable:}como método rápido para borrar todos los datos de una variable podemos hacer:
\begin{lstlisting}[language=Matlab]
>> a = [1 2];
>> a = 0;
\end{lstlisting}

\paragraph{Eliminar una variable:}para eliminar una variable, utilizamos el comando $clear$:
\begin{lstlisting}[language=Matlab]
>> clear a
>> who
j
\end{lstlisting}


\section{Números complejos}

La forma de operar con números complejos es igual que para los reales. Veamos un ejemplo aplicando la formula para obtener las raíces de una función cuadrática.

\begin{lstlisting}[language=Matlab]
>> a=1; b=2; c=3;
>> x1=(-b+sqrt(b^2-4*a*c))/(2*a)
x1 =
-1.0000 + 1.4142i
>> x2=(-b-sqrt(b^2-4*a*c))/(2*a)
x2 =
-1.0000 - 1.4142i
>> a*x1^2+b*x1+c
ans =
0
\end{lstlisting}

\paragraph{Escribir un complejo:}
\begin{lstlisting}[language=Matlab]
>> c1=1+2*i
c1 =
1.0000 + 2.0000i
>> c2=1-2*i
c2 =
1.0000 - 2.0000i
\end{lstlisting}




\section{Operaciones con variables lógicas:}

Las variables lógicas son vectores ó matrices cuyos elementos toman los valores 0 $falso$ ó 1 $verdadero$. Por ejemplo:
\begin{lstlisting}[language=Matlab]
>> v3=[10 4 8];
>> compara = v3>5
compara =
		1 0 1
\end{lstlisting}

La variable $compara$ es un vector que indica que las componentes 1 y 3 cumplen la condición de ser mayores que 5. Si se desease obtener un vector solo con las componentes que cumplen la condición:
\begin{lstlisting}[language=Matlab]
>> v4=v3(v3>5);
>> v4
v4 =
		10 8
\end{lstlisting}

\paragraph{Comparar dos vectores:} si se comparan dos vectores:
\begin{lstlisting}[language=Matlab]
>> v5=[4 5 8];
>> iguales = v3 ==v5
iguales =
		0 0 1
\end{lstlisting}
hay coincidencia en la tercera componente.

\paragraph{Encontrar elementos que cumplan una condición:}las variables lógicas pueden ser usadas para realizar búsquedas dentro de vectores y matrices. El comando $find$ permite buscar dentro de vectores y matrices los elementos que cumplan con alguna condición requerida. Por ejemplo:
\begin{lstlisting}[language=Matlab]
>> x=sin(2*pi/5*(0:4))
x =
		0.00000 0.95106 0.58779 -0.58779 -0.95106
>> find(x<0.5)
ans =
		1 4 5
\end{lstlisting}