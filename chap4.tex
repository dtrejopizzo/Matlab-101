\chapter{Matrices}

\textbf{MATLAB} trabaja esencialmente con matrices de números reales o complejos. Las matrices 1 x 1 son interpretadas como escalares y las matrices fila o columna como vectores. Por defecto todas las variables son matriciales y nos podemos referir a un elemento con dos índices. Aun así, conviene saber que la matriz esta guardada por columnas y que nos podemos referir a un elemento empleando solo un índice, siempre que contemos por columnas. Insistiremos bastante en este detalle, porque tiene fuertes implicaciones para entender el funcionamiento de bastantes aspectos de \textbf{MATLAB}

\section{Definición de matrices y sub-matrices}
Para definir una matriz hay que darle un nombre, y agregar los datos entre corchetes, separando cada valor por un espacio o una coma, y cada final con punto y coma.
\begin{lstlisting}[language=Matlab]

>> A = [1 2 3; 4 5 6; 7 8 9]

A =

     1     2     3
     4     5     6
     7     8     9

\end{lstlisting}

Si queremos acceder a un elemento de alguna matriz que hallamos definido anteriormente, solo debemos escribir el nombre de la matriz, y entre paréntesis la fila y la columna del elemento, como se ve a continuación.
\begin{lstlisting}[language=Matlab]
>> A(1,2)

ans =

     2

\end{lstlisting}

Si queremos acceder a una fila de alguna matriz que hallamos definido anteriormente, debemos escribir $f$ seguido del número de fila, indicar el nombre de la matriz y entre paréntesis, colocar el número de la fila en el campo correspondiente, en el campo de la columna debemos poner $:$.
\begin{lstlisting}[language=Matlab]
>> f1=A(1,:)

f1 =

     1     2     3

\end{lstlisting}

Si queremos acceder a una columna de alguna matriz que hallamos definido anteriormente, debemos escribir $c$ seguido del número de columna, indicar el nombre de la matriz y entre paréntesis, colocar el número de la columna en el campo correspondiente, en el campo de la fila debemos poner $:$.
\begin{lstlisting}[language=Matlab]
>> c1=A(:,1)

c1 =

     1
     4
     7

>> 
\end{lstlisting}

\section{Operaciones}
Con las operaciones suma (+) y producto (*) entre matrices hay que poner atención en que las
dimensiones de las matrices sean las adecuadas para realizar dichas operaciones. Si las operaciones se realizan con matrices de tamaño incompatible se obtiene un mensaje de error.

\paragraph{Suma:} para sumar dos matrices, solo basta con escribir el nombre de las matrices con un signo (+) entre ambas (colocadas en el orden en que se desea realizar la operación).
\begin{lstlisting}[language=Matlab]
>> A+B

ans =

     1     4     6
     5    10    10
    13    15    12

\end{lstlisting}

\paragraph{Resta:} para restar dos matrices, solo basta con escribir el nombre de las matrices con un signo (-) entre ambas (colocadas en el orden en que se desea realizar la operación).
\begin{lstlisting}[language=Matlab]
>> A-B

ans =

     1     0     0
     3     0     2
     1     1     6

\end{lstlisting}

\paragraph{Multiplicación de un escalar por una matriz:} se debe indicar un valor (real), y la matriz que se desea multiplicar, con un signo (*) entre ambos.
\begin{lstlisting}[language=Matlab]
>> 3*A

ans =

     3     6     9
    12    15    18
    21    24    27

\end{lstlisting}

\paragraph{Multiplicación de dos matrices:} se deben indicar las dos matrices, con un signo (*) entre ambas. Si las dos matrices son de igual cantidad de filas y columnas, la matriz resultado sera de igual cantidad de filas y columnas que estas.
\begin{lstlisting}[language=Matlab]
>> A*B

ans =

    20    33    20
    41    75    50
    62   117    80

\end{lstlisting}

En cambio, si las matrices son distintas, se deberá tener en cuenta la cantidad de columnas de la primer matriz y la cantidad de filas de la segunda matriz, si ambos son iguales entonces se puede realizar la multiplicación y la matriz resultado tendrá la cantidad de filas de la primer matriz y la cantidad de columnas de la segunda matriz. Si no son iguales, obtendremos un mensaje de error. Definamos una matriz C, de 3 filas y 2 columnas, y una matriz D de 2 filas y 3 columnas.

\begin{lstlisting}[language=Matlab]
>> C = [2 4; 6 3; 7 8]

C =

     2     4
     6     3
     7     8

>> D = [2 4 5; 2 6 1]

D =

     2     4     5
     2     6     1

\end{lstlisting}

A es una matriz de 3 filas y 3 columnas, por lo tanto, el resultado de $A*C$ sera:
\begin{lstlisting}[language=Matlab]
>> A*C

ans =

    35    34
    80    79
   125   124

\end{lstlisting}

Ahora, si multiplicamos A con D, recibiremos un mensaje de error, debido a que la cantidad de columnas de A no coincide con la cantidad de filas de D.
\begin{lstlisting}[language=Matlab]
>> A*D
??? Error using ==> mtimes
Inner matrix dimensions must agree.
 
\end{lstlisting}

\paragraph{Multiplicar componente a componente:} si deseamos multiplicar componente a componente, debemos escribir el nombre de las dos matrices a multiplicar, escribiendo (.*) entre ambas.
\begin{lstlisting}[language=Matlab]
>> A.*B

ans =

     0     4     9
     4    25    24
    42    56    27

\end{lstlisting}

\paragraph{Potencia de una matriz:} se puede calcular la potencia enésima de una matriz solo si esta es cuadrada y si la potencia es un número entero, debemos escribir la matriz, seguida del símbolo $(^)$y la potencia deseada. Calculemos el cuadrado de la matriz A.
\begin{lstlisting}[language=Matlab]
>> A^2

ans =

    30    36    42
    66    81    96
   102   126   150

\end{lstlisting}

\section{Funciones matriciales}

\subsection{Determinante}

El determinante de una matriz se calcula usando la instrucción $det$. Determinemos una matriz E de 5 filas y 5 columnas, luego calcularemos su determinante.
\begin{lstlisting}[language=Matlab]
>> E = [4 5 2 1 3; 2 2 2 4 2; 8 4 2 1 2; 6 4 4 8 10; 2 0 0 0 2]

E =

     4     5     2     1     3
     2     2     2     4     2
     8     4     2     1     2
     6     4     4     8    10
     2     0     0     0     2

>> det(E)

ans =

  -48.0000

\end{lstlisting}

\subsection{Matriz inversa}

Para calcular la inversa de una matriz, primero debemos asegurarnos de que sea una matriz cuadrada, y luego de que el determinante de esta sea distinto de 0. Calculamos a continuación la inversa de la matriz E definida en el punto anterior.
\begin{lstlisting}[language=Matlab]

>> inv(E)

ans =

    0.0000    0.5000   -0.0000   -0.2500    0.7500
    1.0000    2.5000   -1.0000   -1.2500    3.2500
   -2.3333   -8.0000    3.0000    3.9167  -11.0833
    0.6667    3.0000   -1.0000   -1.3333    3.6667
   -0.0000   -0.5000    0.0000    0.2500   -0.2500

\end{lstlisting}



\subsection{Transpuesta de una matriz}

Para obtener la traspuesta de una matriz, debemos escribir el nombre de la matriz y un apostrofe a continuación. Ejemplo de transposición de la matriz A.
\begin{lstlisting}[language=Matlab]
>> A'

ans =

     1     4     7
     2     5     8
     3     6     9

\end{lstlisting}



\subsection{Rango de una matriz}

Para calcular el rango de una matriz, debemos usar la instrucción $rank()$ y el nombre de la matriz entre paréntesis. A continuación calculamos el rango de la matriz A.
\begin{lstlisting}[language=Matlab]
>> rank(A)

ans =

     2

\end{lstlisting}


\subsection{Traza de una matriz}

Para obtener la traza de una matriz debemos usar la instrucción $trace()$, con el nombre de la matriz entre paréntesis. Como ejemplo, calculamos la traza de la matriz A.
\begin{lstlisting}[language=Matlab]
>> trace(A)

ans =

    15

\end{lstlisting}


\subsection{Polinomio característico y raíces o espectro:}
En álgebra lineal, se asocia un polinomio a cada matriz cuadrada llamado polinomio característico. Dicho polinomio contiene una gran cantidad de información sobre la matriz, los más significativos son los valores propios, su determinante y su traza.


Definamos una matriz F de 2x2 y calculemos su polinomio característico mediante el comando $poly()$.
\begin{lstlisting}[language=Matlab]
>> F = [-17 9; -30 16]

F =

   -17     9
   -30    16

>> p = poly(F)

p =

     1     1    -2

\end{lstlisting}
Este comando nos devuelve los coeficientes del polinomio ordenado y completo, dicho polinomio es: $$p(\lambda) = \lambda^{2} + \lambda - 2$$

Para calcular las raíces del polinomio o espectro, utilizamos:
\begin{lstlisting}[language=Matlab]
>> roots(p)

ans =

    -2
     1
\end{lstlisting}


\subsection{Auto-valores y Auto-vectores}
Siguiendo con la matriz F de 2x2, ahora vamos a calcular sus auto-valores mediante el comando $eig()$ y sus correspondientes auto-vectores. La matriz $evas$ contiene los auto-valores, y la matriz $eves$ contiene los auto-vectores (la primer columna de $eves$ es el auto-vector correspondiente al auto-valor de la primer columna de $evas$, y la segunda columna de $eves$ contiene el auto-vector correspondiente al auto-valor de la segunda columna de $evas$):
\begin{lstlisting}[language=Matlab]
>> eig(F)

ans =

    -2
     1

>> [eves,evas]=eig(F)

eves =

   -0.5145   -0.4472
   -0.8575   -0.8944


evas =

    -2     0
     0     1

\end{lstlisting}


\subsection{Exponenciacion matricial}
Esta operación lo que hace es elevar $e$ por una matriz, en concreto si utilizamos la matriz F estaríamos haciendo lo siguiente: $e^{F}$ (miembro a miembro). Se utiliza el comando $expm()$, el resultado de esta operación es:
\begin{lstlisting}[language=Matlab]
>>  expm(F)

ans =

  -12.7794    7.7488
  -25.8295   15.6330

\end{lstlisting}


\section{Matrices especiales}

\paragraph{Matriz identidad:} para construir una matriz identidad, debemos usar el comando $eye()$ especificando el tamaño de la matriz entre paréntesis. A continuación, generamos la matriz identidad de 3x3.
\begin{lstlisting}[language=Matlab]
>> eye(3)

ans =

     1     0     0
     0     1     0
     0     0     1

\end{lstlisting}

\paragraph{Matriz nula:} para construir una matriz nula de $m~x~n$ filas y columnas, debemos usar el comando $zeros(m,n)$. A continuación, se muestra una matriz nula de 2 filas y 3 columnas.
\begin{lstlisting}[language=Matlab]
>> zeros(2,3)

ans =

     0     0     0
     0     0     0

\end{lstlisting}

\paragraph{Matriz de unos:} para generar una matriz de unos de $m~x~n$ filas y columnas, debemos utilizar el comando $ones(m,n)$. A continuación, se muestra una matriz de unos de 2 filas y 3 columnas.
\begin{lstlisting}[language=Matlab]
>> ones(2,3)

ans =

     1     1     1
     1     1     1

\end{lstlisting}

\paragraph{Matriz diagonal inferior:}
\begin{lstlisting}[language=Matlab]
>> tril(R) % matriz con la parte triangular inferior de R
ans =
1 0
6 9
\end{lstlisting}

\paragraph{Matriz diagonal superior:}
\begin{lstlisting}[language=Matlab]
>> triu(R) % matriz con la parte triangular superior de R
ans =
1 5
0 9
\end{lstlisting}

\section{Matrices sparse (ralas)}

Hay un sistema especial de almacenamiento y manipulación de matrices ralas. La función genera una matriz rala de dimension \textbf{n}x\textbf{m}, cuyos únicos elementos no nulos son los de sub-indices $(i_{k},j_{k})$, de valor \textbf{$c_{k}$}. La sintaxis de la función es:

\begin{lstlisting}[language=Matlab]
>> sparse(i,j,c,m,n)
\end{lstlisting}
donde:
\begin{itemize}
\item i,j: son vectores de subintrares, de la misma longitud.
\item c: es un vector de la misma longitud que los anteriores.
\item n,m: son números naturales.
\end{itemize}

Ejemplo:
\begin{lstlisting}[language=Matlab]
>> fil = [1,1,2,3,4];
>> col = [1,3,2,4,1];
>> val = [1,2,5,3,4];
>> M = sparse(fil,col,val,4,4)

M =

   (1,1)        1
   (4,1)        4
   (2,2)        5
   (1,3)        2
   (3,4)        3
\end{lstlisting}