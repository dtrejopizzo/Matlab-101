\chapter{Vectores}

\section{Operaciones con vectores}

En \textbf{MATLAB} todos los datos son matrices. Los vectores y escalares son casos particulares de matrices. No es necesario especificar previamente las dimensiones de un vector o de una matriz, se definen de forma interactiva al crearse. \textbf{MATLAB} introduce de manera muy natural el manejo de vectores a través del uso de corchetes. Un vector-fila de dimension \textbf{n} se pude definir  escribiendo sus componentes entre corchetes, separándolos por comas o espacios en blanco. Para comenzar, vamos a definir un vector:

\paragraph{Definir un vector:}
\begin{lstlisting}[language=Matlab]
>> a = [1 2 3 4 6 4 3 4 5]
a =
1 2 3 4 6 4 3 4 5
\end{lstlisting}

Otras ordenes para definir vectores son:

\begin{itemize}
\item \textbf{v1 = a:h:b}: define un vector-fila cuyas componentes van desde a hasta un número c<b, en incrementos de h.
\item \textbf{v2 = linspace(a,b,n)}: define un vector de longitud \textbf{n}, partición regular en el intervalo [a,b].
\end{itemize}

\subsection{Operaciones con vectores}

\paragraph{Sumar un escalar a un vector:}
\begin{lstlisting}[language=Matlab]
>> b = a + 2
b =
3 4 5 6 8 6 5 6 7
\end{lstlisting}

\paragraph{Sumar dos vectores:}
\begin{lstlisting}[language=Matlab]
>> c = a + b
c =
4 6 8 10 14 10 8 10 12
\end{lstlisting}

\paragraph{Multiplicar dos vectores:}
\begin{lstlisting}[language=Matlab]
>> d = a .* b
d =
3 8 15 24 48 24 15 24 35
\end{lstlisting}

\paragraph{Trasponer un vector:} se hace por medio del uso de un apostrofe.
\begin{lstlisting}[language=Matlab]
>> d'
d =
	3 
	8 
	15 
	24 
	48 
	24 
	15 
	24 
	35
\end{lstlisting}


\paragraph{Vectores cuyos elementos están igualmente espaciados:}si se quiere introducir un vector cuyos elementos tengan valores igualmente espaciados se usa el operador $:$ (dos puntos). Por ejemplo, queremos definir un vector D cuyas componentes sean 1, 2, 3 y 4, entonces:
\begin{lstlisting}[language=Matlab]
>> D=1:4
\end{lstlisting}

Esto es muy útil para tabular funciones donde es necesario que la variable independiente x tome valores entre un valor inicial y otro final, por ejemplo:
\begin{lstlisting}[language=Matlab]
x=0:0.1:3
\end{lstlisting}

Aquí, el vector v tiene componentes 0, 0.1, 0.2,...,3 lo que es equivalente a que la variable x tome valores desde 0 hasta 3 separados en 0.1, los que quedan almacenados en un vector llamado x.

\paragraph{Acceder a una posición del vector:}para poder tener acceso a una componente de un vector:
\begin{lstlisting}[language=Matlab]
>> x(3)
ans = 0.200
\end{lstlisting}

\paragraph{Extraer sub vectores:} se puede extraer un vector de otro dado de la siguiente forma:
\begin{lstlisting}[language=Matlab]
>> x(3:7)
ans =
	0.2000	0.3000	0.4000	0.5000	0.6000
\end{lstlisting}

\subsection{Resumen de operaciones}

\begin{itemize}
\item Operaciones con vectores y escalares
\begin{itemize}
\item \textbf{v+k}
\item \textbf{v-k}
\item \textbf{k*v}
\item \textbf{v/k}
\item \textbf{k./v}
\item \textbf{$v.^k$}
\item \textbf{$k.^v$}
\end{itemize}

\item Operaciones entre vectores
\begin{itemize}
\item \textbf{v+w}
\item \textbf{v-w}
\item \textbf{v.*w}: producto componente a componente
\item \textbf{c./w}: división componente a componente
\item \textbf{$v.^w$}: exponente componente a componente
\item \textbf{v*w}: si \textbf{v} es un vector-fila de dimension \textbf{n} y \textbf{w} es un vector columna de dimension \textbf{n}, es el producto escalar de v y w.
\end{itemize}
\end{itemize}

\subsection{Funciones especificas de vectores}

Las funciones matematicas lementales admiten vectores como argumentos y se interpretan componente a componente. Algunas funciones son:

\begin{itemize}
\item \textbf{sun(v)}: suma de las componentes del vector v.
\item \textbf{prod(v)}: producto de las componentes del vector v.
\item \textbf{dot(v,w)}: producto escalar de dos vectores del mismo tipo y dimensiones.
\item \textbf{cross(v,w)}: producto vectorial de dos vectores del mismo tipo y dimension 3.
\item \textbf{max(v)}: máximo d las componentes del vector v. No contempla valor absoluto.
\item \textbf{norm(v)}: norma euclidea del vector v.
\item \textbf{norm(v,p)}: norma-p del vector v: $sum(abs(v)^p)^(1/p)$
\item \textbf{norm(v,inf)}: norma infinito del vector v.
\end{itemize}

\section{Operaciones con polinomios}

Los polinomios se representan en \textbf{MATLAB} como vectores fila. Por ejemplo, el polinomio
$3s^3  -5s^2 + 7s + 3$ se representa por

\begin{lstlisting}[language=Matlab]
>> p=[3 -5 7 3]
\end{lstlisting}

\paragraph{Raíces de un polinomio:}las raíces de un polinomio se hallan mediante la función $roots$:
\begin{lstlisting}[language=Matlab]
>> r=roots(p)
\end{lstlisting}

\paragraph{Producto de dos polinomios:}
El producto de dos polinomios se realiza a través de la convolución de los vectores de sus coeficientes, mediante la función $conv$. Por ejemplo:
\begin{lstlisting}[language=Matlab]
>> p1=[-1 -3 3 4];
>> p2=[1 2 4 0];
>> p=conv(p1,p2);
\end{lstlisting}

\paragraph{División de polinomios:}
Para la división se usa la deconvolución. Mediante la función $deconv$ se obtiene el cociente q y el resto r de la división.
\begin{lstlisting}[language=Matlab]
>> [c,r]=deconv(p,p1);
\end{lstlisting}

La función $polyval$ sirve para hallar el valor de un polinomio. Si el parámetro que le pasamos es un vector, calcula otro vector con los valores del polinomio para cada uno de los del vector. La función $polyfit$ sirve para hacer ajustes polinómicos de una secuencia de datos dada por dos vectores X e Y. Se puede elegir el grado del polinomio. En el siguiente ejemplo se utilizan estas dos funciones:
\begin{lstlisting}[language=Matlab]
>> x=[0:10];
>> y=rand(x);
>> plot(x,y)
>> p=polyfit(x,y,3); % Elegimos grado 3
>> z=polyval(p,x);
>> hold
>> plot(x,z)
\end{lstlisting}

Citaremos por ultimo la función $residue$ que sirve para hallar los residuos de una función racional en los polos de la misma (o coeficientes de su expansión en fracciones simples), bajo el supuesto de que los polos sean simples. Además, dicha función calcula también los polos y el termino directo.
\begin{lstlisting}[language=Matlab]
>> B = [1 2 3 4];
>> A = [1 2 3 4 5 6 7];
>> [r,p,k] = residue(B,A)

r =

   0.1502 - 0.0022i
   0.1502 + 0.0022i
  -0.0314 - 0.0725i
  -0.0314 + 0.0725i
  -0.1187 - 0.1531i
  -0.1187 + 0.1531i


p =

  -1.3079 + 0.5933i
  -1.3079 - 0.5933i
  -0.4025 + 1.3417i
  -0.4025 - 1.3417i
   0.7104 + 1.1068i
   0.7104 - 1.1068i

\end{lstlisting}